\documentclass[dissertation.tex]{subfiles}
\begin{document}

The project was a Success. I have implemented tools that are able to reproduce
experiments equivalent to those of Tishby's and Saxe's. I've added a number of
features to the implementation, that help with: Data analysis, Configuration of
experiments, extensibility of code, and performance improvements. Using the
tools, I was able to reproduce Tishby's results and show that they are robust to
changes in some parameters. I've investigated the results published by Saxe and
devised an experiment which contradicts his results. This lead me to doubt the
tools Saxe, Tishby, and I was using. To address this problem My supervisor and
I, devised a method that should improve the performance of our tools --
Batching. However, the method prevents us from answering the original questions
Tishby was asking.

\section{Looking Back}
I have underestimated the importance of subtle details present in the theory of
this topic. Having a subpar understanding, I have made wrong assumptions, which
lead to preventable mistakes and wasted time. I implemented code for mutual
information estimation with out fully understanding the importance of noise in
compression -- this made some of my results invalid.

If I were to reimplement the code again I would first derive a mathematical
model of the important parts of the project, before any code was written. Having
this model would have helped me tremendously.

\section{Further Work}
There is lots of further work that need to be done:
\begin{itemize}
  \item{
      We need better ways to estimate mutual information.
    }
  \item{
      It would be helpful to understand why the Binning and KDE MIE show
      different results for the activation function ReLu.
    }
  \item{
      It would be great to see work being done on NN-specific MI estimation. In
      a similar way to the batching method that assumes knowledge about the
      structure of a NN.
    }
  \item{
      I believe that before we get better tools we cannot ask questions about
      the "phases" a NN is in such as the "Compression Phase" and "Memorization
      Phase" as mentioned in \autoref{subVIZ}.
    }
  \item{
      Lastly, This topic lacks a concrete mathematical backing it's claims. 
    }
\end{itemize}
Overall, I think the topic of compression within NN is a very interesting one with
the possibility to help us understand the inner working of NNs. It is a very
young idea and would contribute greatly with more attention it gets.

\end{document}
