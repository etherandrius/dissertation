
\begin{itemize}
  \item{
      Deep Neural Networks (DNNs) although extremely successful and widely
      adopted are still a black box machine, as there is no comprehensive
      understanding on how they learn.
    }
  \item{
      Recently Prof.\ Tishby produced a paper claiming to understand the basic
      principle of how DNNs work.
    }
  \item{
      He suggests that there are two phases during the training of a DNN - the
      fitting phase and the compression phase. 

      he suggested there are two phases of during the training period: the
      fitting phase and the compression phase
    }
  \item{
      He showed this by visualizing the DNN via an Information Plane
    }
  \item{
      Tishby makes some interesting and significant claims about how DNN`s work,
      however he does not provide a formal proof, his conclusions are only based
      on experimental evidence. In our work we reimplement his experiments as a
      form of independent verification, showing that they are robust and not a
      result of specific parameters. We take a look at a paper by Saxe that
      tries to debunk Tishby`s ideas, and redo some of the experiments presented
      in Saxe`s paper. Lastly experiments that Tishby devised do not capture the
      full idea that he is presenting, as a result we devised a set up that we
      think mirrors Tishby`s ideas more closely.
      
      We also take a look at opposing papers
      focusing mostly on paper by Saxe, 

      In our work we explore the ideas that Tishby presented in his work.
      Tishby`s work has attracted a lot of scepticism we reimplement his
      experiments as a form of independent verification 
      

      we tried to verify his claims experimentally by reproducing his
      experiments, looking at what his opponents say and what experiments they
      conduct have conducted.
    }
  \item{
      thus giving more merit to the experimental evidence
    }
  \item{
      Lastly we have devised experiments that aim to better capture Tishby`s
      ideas.
    }
\end{itemize}

