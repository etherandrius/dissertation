
The thesis is compromised of multiple of experiments some of which are
re-implementations of experiments written by Tishby or Saxe. We aimed to
reproduce the experiment as a feat for independent verification and hopefully to
provide a baseline codebase for other researches to conduct their experiments
and work upon.

% need to explain motivation and experiment layout here 
% will explain what the results showed in the next chapter evaluation
% this is suboptimal
% need to use a lot of forward references to the next chapter

\section{Mutual Information Estimation}


\begin{itemize}
  \item{
      why Mutual Information Estimation is hard
    }
  \item{
      what have I tried
      \begin{itemize}
        \item{
            wgao - tried to implement, but failed the implementation was too
            complex and decided to cut my loses as it was taking too much time
            to implement. Emailed the author, he wasn't able to provide any
            code.
          }
        \item{
            wgao9 (lnn) - local nearest neighbour mutual information estimation.
            Based on a paper1 the code was available online on Github.
            The code produced nonsensical results highly sporadic and often even
            negative results for measure of mutual information. Although the
            code might have been fixable running it took an extremely long time
            so I've decided to not use this method as it already has consumed a
            lot of my time.
          }
      \end{itemize}
    }
  \item{
      what worked
      \begin{itemize}
        \item{
            Tishby - describe how Tishby calculated mutual information, what
            assumptions he made and what faults it has.
          }
        \item{
            Saxe - Saxe used kernel density estimation for to calculate some of
            the experiments this is a more rigorous way to measure information
            but it still produces less than stellar results.
          }
      \end{itemize}
    }
  \item{
      even though the MIE we ended up using is by far not the most sophisticated
      technique, it is still the state of the art that has been used in regards
      to measuring mutual information inside neural networks.
    }
\end{itemize}
    


\section{Tishby`s reproduction}
\section{Saxe`s reproduction}

