\documentclass[dissertation.tex]{subfiles}
\begin{document}

\begin{itemize}
  \item{
      Success Criteria
    }
  \item{
      we have successfully reproduced the results showed by Tishby and Saxe.
      However there are reasons to not trust either of them as they have flaws
      with them.
      \begin{itemize}
        \item{
            Tishby -- used only a toy dataset 
          }
        \item{
            Saxe -- Changed allot of parameters at once made the claim that no
            compression phase happens
          }
      \end{itemize}
    }
  \item{
      We need better tools for MIE
      \begin{itemize}
        \item{
            cannot judge subtleties if something has a compression phase our MIE
            are not trustesd
          }
        \item{
            we have seen KDE and Discrete show inconsistent results, when n'th
            layers hass less information about the input than the n+1'th layer.
          }
      \end{itemize}
    }
  \item{
      <++>
    }
  \item{
      performance
    }
\end{itemize}

---------------------------------------

\subsection{Deterministic networks} \label{ssection:detnet}

There's a very real argument to be made against compression in neural networks.
Consider a generic neural network we can think of it as a function that is a
series of matrix transformation, where a matrix corresponds to weights of a
specific layer. However these matrices are all random (at least at the start of
training) and hence probability of them being invertible is 100\%. 

Knowing that every single matrix is invertible allows us to conclude that that
neural network as a whole is an invertible function, which means no information
is lost and compression is impossible.

% matrix is invertible if it's random, maybe the neural network tends to
% non-invertible matrices ? if this was the case we would see mutual information
% full at the start and it would decrease over time - however we see that mutual
% information increases for Y and decreases for X


\subsection{Why Randomness is hard to capture} \label{ssection:rename}


\end{document}
