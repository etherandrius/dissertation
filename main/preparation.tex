
\begin{itemize}
  \item{
      theoretical basis 
      \begin{itemize}
        \item{
            MIE, what Tishby has actually shown and why is it contentious
          }
        \item{
            Had to fully understand the main points Tishby was making and other
            peoples opinions in the field in order to avoid downfalls that might
            have already been explored.
          
          }
        \item{
            Had to do research MIE`s (Mutual Information Estimator`s) from the
            start we knew that MIE will be a contentious part of the project as
            there has been little work done (to the best of my knowledge) on
            estimating mutual information of continues distributions from
            discrete sample sets
          }
      \end{itemize}
    }
  \item{
      I needed to come up with a plan on how am I gonna reproduce the results
      \begin{itemize}
        \item{
            understand the setup up of Tishby`s experiments exactly how he
            produced the results, thankfully the code was publicly available
            Tishby recently released his results on Github which I used as a
            point of reference.
          }
        \item{
            figure out what technologies I am going to use. For his
            implementation Tishby used the Tensorflow framework. I've decided to
            use the Keras framework as I've found out Keras produced code that
            is more concise and easier to read/maintain and easy to read.
            Further more rewriting the code in a different framework made so
            that I can't rely on the detail of Tishby`s code and avoid mistakes
            that might be in the original code.
          }
      \end{itemize}
      
    }
\end{itemize}

Preparation for the project required me to understand how Tishby arrived to the
conclusion that he has presented us in his paper, I needed to understand how the
experiments in the paper were constructed in order to be able to reproduce them reliably.

I needed to understand 

Finally In order to be able to reproduce the results and extend on the ideas I
needed to learn python and frameworks such as Keras and Tensorflow.
\begin{itemize}
  \item{
      Tensorflow - this is the framework Tishby decided to use for
      his implementation.
    }
  \item{
      Keras - this is the framework I have decided to use for my implementation
      as the produced code is much more concise.
    }
\end{itemize}


