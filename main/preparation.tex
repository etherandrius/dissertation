Before developing a plan for how we are going to realize the project in code we
needed to fully understand the ideas presented in the paper:
\begin{itemize}
    \begin{item}
      We needed to identify the main ideas of the paper and understand why some
      parts of the paper are not agreed upon in the scientific community.
      Understand why his ideas are contentious and whether reproducing his
      experiments could bring more validity to his claims. This involved reading
      papers published by Tishby and academics who shown an opposing view to
      him.
    \end{item}
    \begin{item}
      A main tool that the paper relies on is MIE (Mutual Information
      Estimation). Reading about MIE we quickly understood that MIE is a
      contentious part of the project as a result we had to do a decent amount
      of research regarding the subject. MIE is difficult because we are trying
      to estimate information between two continuous distributions using only a
      discrete sample set. This area has not seen much academic attention so the
      tools we ended up using could be greatly improved in the future.
    \end{item}
\end{itemize}

Once we had a reasonable understanding of the ideas in the paper and which areas
needed more attention we diverted our attention to figuring out the details of
how the experiments were conducted figure out what hyper parameters Tishby
decided are important and what assumptions he made whilst devising the
experiments. 

In addition we needed to find out what resources are available to us online,
what programming frameworks we are going to use for the projects implementation,
and to think about possible extensions to the project once the success criteria
has been achieved.

\begin{itemize}
  \item{
      Online Resources: The two main papers by Tishby and by Saxe have made
      their code public online via Github, we made 

      Online Resources: The two main papers we were looking at has made their
      code available to the public via Github, the papers are Tishby`s paper and
      the main opposing paper by Saxe.
    }
  \item{
      Programming frameworks: The original experiment implementation by Tishby
      has used the Tensorflow framework. We have decided to use the Keras
      framework as it produces code that is more concise and is easier to
      read/maintain. Furthermore rewriting the experiments in a different
      framework means that we cannot rely on the details of Tishby`s and
      potentially avoid any mistakes that may exist in the original
      implementation.
    }
\end{itemize}

\begin{itemize}
  \item{
      theoretical basis 
      \begin{itemize}
        \item{
            MIE, what Tishby has actually shown and why is it contentious
          }
        \item{
            Had to fully understand the main points Tishby was making and other
            peoples opinions in the field in order to avoid downfalls that might
            have already been explored.
          
          }
        \item{
            Had to do research MIE`s (Mutual Information Estimator`s) from the
            start we knew that MIE will be a contentious part of the project as
            there has been little work done (to the best of my knowledge) on
            estimating mutual information of continues distributions from
            discrete sample sets
          }
      \end{itemize}
    }
  \item{
      I needed to come up with a plan on how am I gonna reproduce the results
      \begin{itemize}
        \item{
            understand the setup up of Tishby`s experiments exactly how he
            produced the results, thankfully the code was publicly available
            Tishby recently released his results on Github which I used as a
            point of reference.
          }
        \item{
            figure out what technologies I am going to use. For his
            implementation Tishby used the Tensorflow framework. I've decided to
            use the Keras framework as I've found out Keras produced code that
            is more concise and easier to read/maintain and easy to read.
            Further more rewriting the code in a different framework made so
            that I can't rely on the detail of Tishby`s code and avoid mistakes
            that might be in the original code.
          }
      \end{itemize}
      
    }
  \item{
      starting point
      \begin{itemize}
        \item{
            Had Tishby`s paper and code as a reference point of what has to be
            achieved 
          }
        \item{
            ended up following other's peoples implementations of MIE`s closely
            some of which didn't work and had to be scrapped.
          }
      \end{itemize}
    }
\end{itemize}

Preparation for the project required me to understand how Tishby arrived to the
conclusion that he has presented us in his paper, I needed to understand how the
experiments in the paper were constructed in order to be able to reproduce them reliably.

I needed to understand 

Finally In order to be able to reproduce the results and extend on the ideas I
needed to learn python and frameworks such as Keras and Tensorflow.
\begin{itemize}
  \item{
      Tensorflow - this is the framework Tishby decided to use for
      his implementation.
    }
  \item{
      Keras - this is the framework I have decided to use for my implementation
      as the produced code is much more concise.
    }
\end{itemize}


