\documentclass[12pt]{article}

\usepackage{minted}
\usepackage{hyperref}
\usepackage{datetime}
\usepackage{datenumber}
\usepackage{advdate}
\usepackage[super]{nth}
\usepackage[margin=0.75in]{geometry}

\parindent 0pt \parskip 6pt

\newcommand\todo[1]{\textbf{TODO(kc506): #1}}
\newcommand\haskell[1]{\mintinline{haskell}{#1}}
\newcommand\monospace[1]{\mintinline{text}{#1}}

\begin{document}

\thispagestyle{empty}

\centerline{\large Computer Science Tripos: Part II Project Proposal}
\vspace{0.4in}
\centerline{\Large\bf Measuring mutual information within Neural networks}
\vspace{0.3in}

\centerline{Andrius Grabauskas, ag939}
\centerline{Robinson College}

% TODO(ag939)
\centerline{\large \textbf{date}}

\vspace{1in}

\begin{tabular}{ p{4cm} p{4.5cm} l }
{\bf Project Originator:} & Andrius Grabauskas & \\[3mm]
{\bf Project Supervisor:} & Dr.\ Damon Wischik & {\bf Signature:} \\[3mm]
{\bf Director of Studies:} & Prof.\ Alan Mycroft & {\bf Signature:} \\[3mm]
{\bf Overseers:} & Dr.\ Robert Mullins & {\bf Signature:} \\[3mm]
                 & Prof.\ Pietro Lio' & {\bf Signature:} \\[3mm]
\end{tabular}

\vspace{0.75in}

\section*{Introduction and Description of the Work}

The goal of this project is to confirm or deny the results produced by
Shwartz-ziv \& Tishby in their paper ``Opening the black box of Deep Neural
Networks via Information``\footnote{https://arxiv.org/abs/1703.00810}

The paper tackles our understating of Deep Neural Networks (DNNs). As of yet
there is no comprehensive theoretical understanding of how DNNs learn from data.
The authors proposed to measure how information travels within the DNNs layers.

They found that training of neural networks can be split into to two distinct
phases: memorization followed by the compression phase.
\begin{itemize}
  \item memorization - each layer increases information about the input and the
    label
  \item compression  - this is the generalization stage where each layer tries
    to forget details about the input while still increasing mutual information
    with the label thus improving performance of the DNN. This phase takes the
    wast majority of the training time.
\end{itemize}

They found that each layer in neural network tries to throw out unnecessary data
from the input while preserving information about the output/label. As the
network is trained each layer preserves more information about the label

The results they found were interesting but also contentious as they have not
yet provided a formal proof, just experimental data as a result there are many
peers that are cautious and sceptical of the theory.

\section*{Starting Point}

I have watched a talk that Prof.\ Tishby gave on this topic at Yandex, no other
preparation was done. 

\section*{Resources Required}

I will use my own laptop to train and evaluate the neural networks as I expect
them to be small given the difficulty of measuring mutual information and
entropy in neural networks.

For backups I intend to store my work on GitHub and my own personal machine. In
case my laptop breaks I will get another one or use the MCS machines.

\section*{Substance and Structure of the Project}

The aim of this project to reproduce the results provided by Prof.\ Tishby and
his colleagues. The intention of my work is to help settle the debate surrounding the
topic either strengthening the arguments in favour of the theory in case my
results are inline with the aforementioned results or encourage discussion in
case my results contradict the theory.

My work will require me to learn a great deal of Information theory and DNN
theory. In order for my results to be valid.

One of the more contentious parts of my project will be measuring mutual
information between the input a layer in the DNN and the label. Mutual
information is known to be tricky to measure when it comes to Continuous random
variables.

Will need to use Python to train the neural networks and GNUplot or alternative
to plot the results.

\section*{Success Criteria}

Confirm or deny the results produced in``Opening the black box of Deep Neural
Networks via Information`` paper on the same dataset as the paper. In order to do
that I will need to:

\begin{itemize}
  \item Train a neural network
  \item Measure the mutual information between input the layers and the label.
  \item Analyze the results
\end{itemize}

\section*{Extensions}

The dataset used in the original paper is very non-standard so it would be very
interesting to see how the theory holds up when we use it with different
datasets.

Different datasets will most likely need different ways to approximate mutual
information between layers.

\section*{Schedule}

% Set start date to 15th October 2018
\ThisYear{2018}\ThisMonth{10}\ThisDay{15}

% Specify date format like "15th Oct"
\newdateformat{datefmt}{\nth{\THEDAY} \shortmonthname[\THEMONTH]}

% Force advdate commands to change the global "today" instance, not the local one
\makeatletter
\renewcommand\AdvanceDate[1][\@ne]{\global\advance\day#1 \FixDate}
\renewcommand\FixDate{%
  \FixMonth \is@LeapYear
  \l@@p \global\ifnum\day<1 \Pr@vD@y \repeat
  \l@@p \M@s\m@sic \global\ifnum\day>\M@s \N@xtD@y \repeat
}
\renewcommand\FixMonth{%
  \L@@p \global\ifnum\month<1 \global\advance\year\m@ne \global\advance\month12 \is@LeapYear \repeat
  \L@@p \global\ifnum\month>12 \global\advance\year\@ne \global\advance\month-12 \is@LeapYear \repeat}
\def\Pr@vD@y{%
  \global\ifnum\day<-366
    \global\ifnum\month>2
      \global\advance\day\r@k \global\advance\year\m@ne \is@LeapYear
    \else
      \global\advance\year\m@ne \is@LeapYear \advance\day\r@k
    \fi
  \else
    \global\advance\month\m@ne \FixMonth
    \global\advance\day\m@sic
  \fi}
\def\N@xtD@y{%
  \global\ifnum\day>366
    \global\ifnum\month>2
      \global\advance\year\@ne \is@LeapYear \global\advance\day-\r@k
    \else
      \global\advance\day-\r@k \global\advance\year\@ne \is@LeapYear
    \fi
  \else
    \global\advance\day-\M@s \global\advance\month\@ne \FixMonth
  \fi}
\makeatother

% Output eg. boldface "15th Oct - 29th Oct" when called.
\newcommand{\daterange}[1]{%
    \textbf{\datefmt\today}
    \textbf{--}
    \AdvanceDate[#1]\relax
    \AdvanceDate[-1]\relax
    \textbf{\datefmt\today}
    \AdvanceDate[1]\relax
}

\begin{itemize}
  \item {
      \daterange{42}

      Preliminary reading and examining code written provided by Prof.\ Tishby
  } \item {
      \daterange{28}

      Reproduce the Papers results on the original dataset.
  } \item {
      \daterange{14}

      Examine result and produce graphs.
  } \item {
      \daterange{42}

      Test the theory on additional datasets 
  } \item {
      \daterange{28}

      Final write up of the thesis
  }
\end{itemize}

\end{document}
